\documentclass[11pt]{article}

\usepackage{fullpage}
\usepackage{amsmath,amssymb,amsthm,amsfonts,latexsym,bbm,xspace,graphicx,float,mathtools,
verbatim, xcolor} 
\PassOptionsToPackage{hyphens}{url}\usepackage{hyperref}
\newcommand{\new}[1]{\textcolor{red}{#1}}
%\usepackage{psfig}

\newcommand{\future}[1]{\textcolor{red}{#1}}

\newcommand{\hP}{\hat P}
\newcommand{\hp}{\hat p}

\newcommand{\Dk}{\Delta_k}
\newcommand{\Px}{P(x)}
\newcommand{\Qx}{Q(x)}
\newcommand{\Nx}{N_x}

\newcommand{\Py}{P(y)}
\newcommand{\Qy}{Q(y)}
\newcommand{\Pml}{P_{ML}}
\newcommand{\Pmlx}{\Pml(x)}
\newcommand{\Pbeta}{P_{\beta}}
\newcommand{\Pbetax}{\Pbeta(x)}


\newcommand{\dTV}[2]{d_{TV} (#1,#2)}
\newcommand{\dKL}[2]{D(#1||#2)}
\newcommand{\chisq}[2]{\chi^2(#1,#2)}
\newcommand{\eps}{\varepsilon}

\newcommand{\nPepsp}[1]{n^*(#1, \eps)}
\newcommand{\nPeps}{\nPepsp{\cP}}


\newcommand{\sumX}{\sum_{x\in\cX}}

\newcommand{\Bpr}[1]{Bern(#1)}

\newenvironment{problem}[2][Problem]{\begin{trivlist}
\item[\hskip \labelsep {\bfseries #1}\hskip \labelsep {\bfseries #2.}]}{\end{trivlist}}

% Theorem-like environments

\newtheorem{Theorem}{Theorem}
\newtheorem{Theorem*}{Theorem}

\newtheorem{Claim}[Theorem]{Claim}
\newtheorem{Claim*}[Theorem]{Claim}
\newtheorem{Corollary}[Theorem]{Corollary}
\newtheorem{Conjecture}[Theorem]{Conjecture}
\newtheorem{CounterExample*}{$\overline{\hbox{\bf Example}}$}
\newtheorem{Definition}[Theorem]{Definition}
\newtheorem{Example}[Theorem]{Example}
\newtheorem{Example*}[Theorem]{Example}
\newtheorem{Exercise}[Theorem]{Exercise}
\newtheorem{Intuition*}[Theorem]{Intuition}
\newtheorem{Joke*}[Theorem]{Joke}
\newtheorem{Lemma}[Theorem]{Lemma}
\newtheorem{Lemma*}[Theorem]{Lemma}
\newtheorem{Open problem}[Theorem]{Open problem}
\newtheorem{Proposition}[Theorem]{Proposition}
\newtheorem{Property}[Theorem]{Property}
\newtheorem{Question}[Theorem]{Question}
\newtheorem{Question*}[Theorem]{Question}
\newtheorem{Remark}[Theorem]{Remark}
\newtheorem{Result}[Theorem]{Result}
\newtheorem{Fact}[Theorem]{Fact}
\newtheorem{Condition}[Theorem]{Condition}

\newcommand{\ed}{\stackrel{\mathrm{def}}{=}}
%\newcommand{\edef}{\stackrel{\mathrm{def}}{=}}
\def \Paren#1{{\left({#1}\right)}}

\newcommand{\probof}[1]{\Pr\Paren{#1}}


\newtheorem{theorem}{Theorem}
\newtheorem{proposition}[theorem]{Proposition}
\newtheorem{corollary}[theorem]{Corollary}
%\newtheorem*{corollary*}{Corollary}
\newtheorem{assumption}[theorem]{Assumption}
\newtheorem{lemma}[theorem]{Lemma}
%\newtheorem*{lemma*}{Lemma}
\newtheorem{conjecture}[theorem]{Conjecture}
\newtheorem{example}[theorem]{Example}
\newtheorem{definition}[theorem]{Definition}
\newtheorem{claim}[theorem]{Claim}

\newcommand{\Xon}{X_1^n}


\newcommand{\prob}{{\rm Pr}}
\newcommand{\Probof}[1]{\prob\left(#1\right)}



\newcommand{\ignore}[1]{}

% Equation formatting

\newcommand{\spreqn}[1]{{\qquad\text{#1}\qquad}}

% Blackboard fonts
\newcommand{\II}{\mathbb{I}} % Added by Theertha on April 16th 2013.
\newcommand{\EE}{\mathbb{E}}
\newcommand{\CC}{\mathbb{C}}
\newcommand{\NN}{\mathbb{N}}
\newcommand{\QQ}{\mathbb{Q}}
\newcommand{\RR}{\mathbb{R}}
\newcommand{\ZZ}{\mathbb{Z}}
\newcommand{\PP}{\mathbb{P}}

% Number sets

\newcommand{\complex}{\CC}
\newcommand{\integers}{\ZZ}
\newcommand{\naturals}{\NN}
\newcommand{\positives}{\PP}
\newcommand{\rationals}{\QQ}
\newcommand{\reals}{\RR}

\newcommand{\realsge}{{\reals_{\ge}}}
\newcommand{\realsp}{\reals^+}
\newcommand{\integersp}{\integers^+}
\newcommand{\integerss}[1]{\integers_{\ge{#1}}}

% boldface

\def \ba     {{\bf a}}
\def \bx     {{\bf x}}
\def \by     {{\bf y}}

\def \bA     {{\bf A}}
\def \bB     {{\bf B}}
\def \bC     {{\bf C}}
\def \bD     {{\bf D}}
\def \bF     {{\bf F}}
\def \bG     {{\bf G}}
\def \bL     {{\bf L}}
\def \bQ     {{\bf Q}}
\def \bR     {{\bf R}}
\def \bS     {{\bf S}}
\def \bT     {{\bf T}}
\def \bX     {{\bf X}}
\def \bY     {{\bf Y}}
\def \bZ     {{\bf Z}}

% caligraphics

\def \cA     {{\cal A}}
\def \cB     {{\cal B}}
\def \cC     {{\cal C}}
\def \cD     {{\cal D}}
\def \cE     {{\cal E}}
\def \cF     {{\cal F}}
\def \cG     {{\cal G}}
\def \cH     {{\cal H}}
\def \cI     {{\cal I}}
\def \cK     {{\cal K}}
\def \cL     {{\cal L}}
\def \cM     {{\cal M}}
\def \cN     {{\cal N}}
\def \cO     {{\cal O}}
\def \cP     {{\cal P}}
\def \cQ     {{\cal Q}}
\def \cR     {{\cal R}}
\def \cS     {{\cal S}}
\def \cT     {{\cal T}}
\def \cU     {{\cal U}}
\def \cV     {{\cal V}}
\def \cW     {{\cal W}}
\def \cX     {{\cal X}}
\def \cY     {{\cal Y}}
\def \cZ     {{\cal Z}}

% vectors

\def \vct#1{{\overline{#1}}}

\def \vcta  {{\vct a}}
\def \vctb  {{\vct b}}
\def \vctq  {{\vct q}}
\def \vcts  {{\vct s}}
\def \vctu  {{\vct u}}
\def \vctv  {{\vct v}}
\def \vctx  {{\vct x}}
\def \vcty  {{\vct y}}
\def \vctz  {{\vct z}}
\def \vctp  {{\vct p}}

\def \vctV  {{\vct V}}
\def \vctX  {{\vct X}}
\def \vctY  {{\vct Y}}
\def \vctZ  {{\vct Z}}

\def \vctbeta  {{\vct\beta}}



% qed's --  Also consider \qedhere

\def \eqed    {\eqno{\qed}}
\def \rqed    {\hbox{}~\hfill~$\qed$}

% sequences

\def \upto  {{,}\ldots{,}}

\def \zn    {0\upto n}
\def \znmo  {0\upto n-1}
\def \znpo  {0\upto n+1}
\def \ztnmo {0\upto 2^n-1}
\def \ok    {1\upto k}
\def \on    {1\upto n}
\def \onmo  {1\upto n-1}
\def \onpo  {1\upto n+1}

% sets

\def \sets#1{{\{#1\}}}
\def \Sets#1{{\left\{#1\right\}}}

\def \set#1#2{{\sets{{#1}\upto{#2}}}}

\def \setpmo   {\sets{\pm 1}}
\def \setmpo   {\{-1{,}1\}}
\def \setzo    {\{0{,}1\}}
\def \setzn    {\{\zn\}}
\def \setznmo  {\{\znmo\}}
\def \setztnmo {\{\ztnmo\}}
\def \setok    {\{\ok\}}
\def \seton    {\{\on\}}
\def \setonmo  {\{\onmo\}}
\def \setzon   {\setzo^n}
\def \setzos   {\setzo^*}


% Set operations



\def\ignore#1{}

 

\title{Assignment Eleven\\ ECE 4200}
\date{}

\begin{document}
\maketitle 

\begin{itemize}
\item
Provide credit to \textbf{any sources} other than the course staff that helped you solve the problems. This includes \textbf{all students} you talked to regarding the problems. 	
\item
You can look up definitions/basics online (e.g., wikipedia, stack-exchange, etc)
\item
{\bf The due date is 12/13/2020, 23.59.59 Eastern time}. 
\item
Submission rules are the same as previous assignments.
\end{itemize}



\begin{problem}{1 (30 points)}
Consider the following movie rating matrix with five users. 

\begin{center}
  \begin{tabular}{ |c |c | c | c | c | c | c| }
    \hline
       & LOTR  & HPATPOZ & Snatch & LSATSB & The Gentlemen & The Hobbit \\ \hline
     A & 5&  \textcolor{red}{?} & 1 & 2 & 3 & 4 \\ \hline            
     B & 5&  4 & 2 & 2 & 2 & 5 \\ \hline 
     C & 1 & 2 & 4 & \textcolor{red}{?} & 4 & 3\\ \hline 
     D & \textcolor{red}{?} & 2 & 3 & 5 & \textcolor{red}{?} &\textcolor{red}{?}\\ \hline 
     E & \textcolor{red}{?} & 3 & 5 & 4 & 5 & 1\\ \hline 
  \end{tabular}
\end{center}

\begin{enumerate}
\item 
Compute the user-user similarity for all the 10 pairs of users and 15 pairs of movies using Pearson's similarity. (Ignore missing values when computing similarity, i.e., you only need to consider the \emph{commonly rated} entries, which can be a lower-dimensional vector).
\item
Let $k=3$. Fill the missing entries of the matrix above using $k$-NN user-user CF and Pearson's similarity. When predicting, use the following formula:
\[
	\hat{r}_{u,i} = \bar{r}_u + \frac{\sum_{j \in K_u} S(u, u_j) (r_{u_j,i} - \bar{r}_{u_j})}{\sum_{j \in K_u} |S(u, u_j)| },
\]
where $\bar{r}_u$ is the average rating of user $u$ (on the items that they actually have rated) and $K_u$ is the top neighbours of $u$ who also rated item $i$. \textbf{Note here we rank user similarities by the absolute values of $S(u, u_j)$} since $S(u, u_j)$ can be negative for pearson similarity. (If I always like things you hate, then your rating is also very useful to me.) 
%
%\item
%Fill in the missing entries with the answer you obtained in the previous part with $k=3$. Perform $k$-means clustering using regular Eucliedean distance for $k=2$ to obtain a cluster of the movies. You can start with an initial cluster centers as LOTR and HPATPOZ.
%\item
%Using the inverse Euclidean similarity distance, perform a hierarchical clustering using single-linkage clustering to generate a hierarchy of clusters for both the movies and the users. 
\end{enumerate}
\end{problem}

\begin{problem}{2 (30 points)}
Consider a Markov chain with three states, Overcast, Rain, and Sunny. The transition probabilities are given in the following table. The $(i,j)th$ entry of the matrix is the probability that the next day to be $j$ if today is $i$. November 29, 2020 is Rain.
\begin{center}
  \begin{tabular}{ |c |c | c | c |}
    \hline
       & O  & S & R \\ \hline
     O & 1/3 & 1/3 & 1/3  \\ \hline            
     S & 1/4 &  1/2 & 1/4  \\ \hline 
     R & 1/4 & 1/4 & 1/2 \\ \hline 
  \end{tabular}
\end{center}
\begin{enumerate}
\item
Draw the state transition diagram with arrows annotating the transition probabilities. 
\item
What is the probability that it will be Sunny on November 30th, 2020? 
\item 
What is the probability that it will Rain on December 2nd, 2020?
\item
What is the probability that it will Rain every day until December 5, 2020 (including it)? 
\item
Compute the probability that it will Rain on December 6, 2020?
\end{enumerate}

\end{problem}

\begin{problem}{3 (30 points)}
	See attached Jupyter Notebooks for details.
\end{problem}



\end{document}
